% !TEX encoding = ISO-8859-16
\section{Radon Wasserstein Barycenters}
\label{sec-bary-radon}

Proposition~\ref{prop-bary-1d} shows that it is computationally inexpensive to compute the Wasserstein barycenter of 1-D densities. It thus makes sense to seek for alternate definitions of barycenters of measures in $\RR^d$ that rely on 1-D Wasserstein distances and barycenters. This section investigates a construction based on the Radon transform.

%%%%%%%%%%%%%%%%%%%%%%%%%%%%%%%%%%%%%%%%%%%%%%%%%%%%%%%%
\subsection{Radon Transform of Functions}

We recall below classical definitions, and refer to~\cite{Helgason-radonbook} for more details. The Radon transform is first defined on integrable functions. 

\begin{defn}[Radon transform of functions]
The Radon transform $Rf$ of $f \in \Lun(\RR^d)$ is defined as
\eql{\label{eq-radon}
		Rf(t,\th) = \int_{\RR^{d-1} } f(t\theta + U_\th \ga) \d \ga
}
where $U_\th \in \R^{d\times(d-1)}$ is any matrix such that its  columns defines an orthogonal basis of $\th^\perp$ (the hyperplane orthogonal to $\th$). 
%  such that $\Theta = [\th, U_\th] \in \R^{d\times d}$ is an orthonormal basis of $\R^d$, {i.e.} the columns of $U_\th$ span the hyperplane $\Hh_{0,\th^{\perp}}$ which passes through the origin and is normal to $\th$.  
This defines $R : \Lun(\RR^d) \rightarrow \Lun(\Om^d)$. 
\end{defn}


\if 0
%% OLD DEF %%%
\eq{
	\foralls (t,\th) \in \Om^d, \quad  
	Rf(t,\th) =
	\int_{ \Hh_{t,\th} } f \d m_{\Hh_{t,\th}} 
	%= {\color{red} \int_{\ga \in \R^{d-1} } f(t\theta + U_\th \ga) d\ga}
	% \int_{ \dotp{x}{\th}=t } f(x) \d x 
}
\eq{
	\qwhereq
	\Hh_{t,\th} = \enscond{x \in \RR^d}{ \dotp{x}{\th}=t }
}
and $m_{\Hh_{t,\th}}$ is the Lebesgues measure on the hyperplane $\Hh_{t,\th}$.
\fi


Its adjoint is defined on continuous functions as follows.

\begin{defn}[Back-projection operator] 
The back projection $R^*g$ of $g \in \Cont{\Om^d}$ is defined as
\eq{
	R^* g(x) % = \int_{ \dotp{x}{\th}=t } g(t,\th) \d t \d \th
	 = \int_{ \Sph } g(\dotp{x}{\th},\th) \d \th.
}
This defines $R^* : \Cont{\Om^d} \rightarrow \Cont{\RR^d}$. 
\end{defn}

\newcommand{\filt}{h}

One has that $R^* R$ is a translation invariant operator, i.e. a convolution  
\eq{
	R^*R f = \filt \star f
	\qwhereq
	\hat\filt(\om) = c \, \norm{\om}^{-(d-1)},
}
where $\star$ is the convolution on $\RR^d$ and $c \in \RR$ is a normalizing constant whose exact value depends on the dimension (see~\cite{Helgason-radonbook}). This relationship suggests a definition of a pseudo-inverse transform which operates on smooth functions so as to invert the low pass filter $\filt$.

\begin{defn}[Inverse Radon transform of functions] 
The pseudo-inverse Radon transform $R^+ g$ of $g \in \Dd(\Om^d)$ is defined as
\eql{\label{eq-pseudo-inv}
	R^{+} g = \filt^+ \star (R^* g)
}
where $\filt^+$ is defined through $\hat \filt^+(\om) = c^{-1} \norm{\om}^{d-1}$.
\end{defn}



%%%%%%%%%%%%%%%%%%%%%%%%%%%%%%%%%%%%%%%%%%%%%%%%%%%%%%%%
\subsection{Radon Transform of Measures}

Since $R^*$ is defined on $\Cont{\RR^d}$, the Radon transform is naturally extended to measures $\mu \in \Mm(\RR^d)$ by duality as follows.

\begin{defn}[Radon transform of measures]\label{defn-radon-measure} For all $\mu \in \Mm(\RR^d)$, we set $\nu=R(\mu)$ be defined through, $\foralls g \in \Cont{\Om^d}$, 
\eql{\label{eq-radon-measure} 
	\int_{\Om^d} g(t,\th) \d\nu(t,\th)
	= \int_{\RR^d} (R^*g)(x) \d \mu(x).
} 
This defines $R : \Mm(\RR^d) \rightarrow \Mm(\Om^d)$.
\end{defn}

The following proposition shows that the Radon transform of a measure gathers projections of the input measure along all possible directions.

\begin{prop}\label{prop-radon-pushforward}
For $\mu \in \Mm(\RR^d)$, one has 
\eq{
	\foralls \th \in \Sph, \quad
	R(\mu)^{\th} = P_\th \sharp \mu
}
\eq{
	\qwhereq
	P_\th : x \in \RR^d \mapsto \dotp{x}{\th} \in \RR,
}
and where $R(\mu)^{\th} \in \Mm(\RR)$ is defined in~\eqref{eq-desintegration}.
\end{prop}

The proof of this proposition, as well as all the other proofs of this section, can be found in Appendix~\ref{sec-appendix-radon}.


The conditional measure $\nu^\th$ associated to $\nu \in \Mm(\Om^d)$ is defined for almost all $\th$, i.e. on a Borel set of $\th \in \Sph$ of measure 1. Proposition~\ref{prop-radon-pushforward} shows that when $\nu=R(\mu)$, then $\nu^\th$ is in fact well defined for all $\th \in \Sph$, because it is a push-forward measure. 


\if 0
Since $R$ is defined on $\Lun(\RR^d)$ and thus on $\Cont(\RR^d)$, it is possible to define $R^*$ on measures by duality. 

\begin{defn}[Adjoint Radon transform of measures]\label{defn-adj-radon-measure} For all $\nu \in \Mm(\Om^d)$, we set $\mu=R^*(\nu)$ be defined through, $\foralls f \in \Cont{\RR^d}$, 
\eql{\label{eq-radon-measure} 
	\int_{\Om^d} g(t,\th) \d\nu(t,\th)
	= \int_{\RR^d} (R^*g)(x) \d \mu(x).
} 
This defines $R : \Mm(\RR^d) \rightarrow \Mm(\Om^d)$.
\end{defn}
\fi


We define in a way similar to Definition~\ref{defn-radon-measure} the inverse Radon transform using the operator $R^{+,*} = R (R^*R)^{-1}$.

\begin{defn}[Inverse Radon transform of measures] For all $\nu \in \Mm(\Om^d)$, we set $\mu=R^+(\nu) \in \Dd^*(\RR^d)$ be defined through, $\foralls f \in \Dd(\RR^d)$, 
\eql{\label{eq-inv-radon-measure}
	\int_{\RR^d} f(x) \d\mu(x)
	= \int_{\Om^d} (R^{+,*}f)(t,\th) \d \nu(t,\th).
}
This defines $R^+ : \Mm(\Om^d) \rightarrow \Dd^*(\RR^d)$. 
\end{defn}

Note that for an arbitrary $\nu \in \Mm(\Om^d)$ (i.e. not necessarily in the range $\Im(R)$ of $R$), $R^+\nu$ is a distribution and not necessarily a measure. One can however show that for $\nu = R(\mu) \in \Im(R)$, then $R^+(\nu)=\mu \in \Mm(\RR^d)$ is a measure, as detailed in the following proposition. The proof of this proposition can be found in~\cite{Boman-Radon}, Section~3.
%  for more details and connections with the celebrated Cram\`er-Wold Theorem.


\begin{prop}\label{prop-im-radon}
	$R : \Mm(\RR^d) \rightarrow \Mm(\Om^d)$ defined in~\eqref{eq-radon-measure} is injective, and $R^+R=\Id_{\Mm(\RR^d)}$.
\end{prop}


The following lemma recapitulates useful commutation properties of the Radon transform with respect to translation and scaling.

\begin{lem}\label{lem-invariances}
		One has, for $\mu \in \Mm_1^+(\RR^d)$ and $\nu \in \Mm_1^+(\Om^d)$, and for all 
		$(s,u,\Phi) \in \RR^{+,*} \times \RR^d \times \Oo(\RR^d)$, 
		\begin{align}
		  	\label{propR} R(\phi_{s,u}\sharp\mu) &= \psi_{s,u}\sharp R(\mu)  \\ 
%		  	\label{propRm} R^*(\psi_{s,u}\sharp\nu) &=  s^{d-1} \cdot (\phi_{s,u}\sharp R^*(\nu)) \\
		  	\label{propRp} R^+(\psi_{s,u}\sharp\nu) &=  \phi_{s,u}\sharp R^+(\nu) \\
			\label{PropRot} R( \Phi \sharp \mu ) &= \tilde\Phi \sharp R(\mu).
		\end{align}			
\end{lem}

%%%%%%%%%%%%%%%%%%%%%%%%%%%%%%%%%%%%%%%%%%%%%%%%%%%%%%%%
\subsection{Radon Barycenter}

According to Proposition~\ref{prop-im-radon}, one has 
\eq{
	R : \Mm_1^+(\RR^d) \rightarrow R(\Mm_1^+(\RR^d)) \subset \bar\Mm_1^+(\Om^d),  
}
although the inclusion on the right hand side is not an equality. This property allows us to define the Radon barycenter.

\begin{defn}[Radon barycenter]\label{defn-radon-baryc} Given $\la \in \La_I$ and $(\mu_i)_{i \in I} \in \Mm_1^+(\RR^d)^I$, we define
\eq{
	\Bary{\RR^d}^R(\mu_i,\la_i)_{i \in I} = 
	R^+ \Bary{\Om^d}^W(R(\mu_i),\la_i)_{i \in I} \in \Dd^*(\RR^d).
}
\end{defn}

Since for $\nu \in \Bary{\Om^d}^W(R(\mu_i),\la_i)_{i \in I}$ one does not have in general $\nu \in \Im(R)$, $\Bary{\RR^d}^R(\mu_i,\la_i)_{i \in I}$ is composed of distributions and not necessarily measures.


The following proposition shows that the Radon barycenter enjoys the same invariance properties to scaling, translation and rotation as the classical Wasserstein barycenter. 


\begin{prop}
\label{prop:InvarianceHolds}
	Proposition~\ref{prop-invariance-W} holds when replacing $\Bary{\RR^d}^W$ by $\Bary{\RR^d}^R$.
\end{prop}



The following proposition shows that, similarly to the usual Wassertstein barycenter, the Radon barycenter of translated and scaled copies of a given measure is also a translated and scaled copy.

\begin{prop}
\label{prop:InvarianceHoldsBis}
	Proposition~\ref{prop-invariance-W-bis} holds when replacing $\Bary{\RR^d}^W$ by $\Bary{\RR^d}^R$.
\end{prop}


%%%%%%%%%%%%%%%%%%%%%%%%%%%%%%%%%%%%%%%%%%%%%%%%%%%%%%%%
\subsection{Approximate Computation with Eulerian Discretization}
\label{subsec-algorithm-eulerian}

% !TEX encoding = ISO-8859-16
\newcommand{\tGg}{{\tilde \Gg}}
\newcommand{\DiscMeasX}[2]{ {m_{#1}^{#2}} }

%% In the following, we consider the special case $d=2$, investigated in Sec.~\ref{sec:results}, without loss of generality. => pas la peine, ce n'est pas juste pour d=2


%%%
\paragraph{Discretization grids.}

We consider here an Eulerian discretization of the Radon barycenter. This means that the considered measures in $\RR^d$ are assumed to be discrete measures supported on the same grid of $N = n^d$ points in $\RR^d$ 
\eq{
	\Gg = \{-n/2+1,\ldots,n/2\}^d
}
(we assume for simplicity that $n$ is even). Similarly, measures on $\Om^d$ are also supported on a fixed grid
\eq{
	\tGg = \Tt \times \Th = 
	\enscond{ (t,\th) }{ t \in \Tt \qandq \th \in \Th}
} 
where $\Tt \subset \RR$ and $\Th \subset (-\pi,\pi]$ are finite sets.

%%%
\paragraph{Measures on grids. }

If $X$ is a discrete set (which in the following will be either $\Gg$, $\tGg$ or $\Tt$), we denote
\eql{\label{eq-gridding}
	\foralls a \in \RR^X, \quad \DiscMeasX{a}{X} = \sum_{x \in X} a_x \de_x \in \Mm_1^+(X).
}
Following the notation introduced in~\eqref{eq-defn-simplex}, we denote $\La_X$ the set of normalized vectors
\eq{
	\La_X = \enscond{ a \in \RR^X }{ \foralls x \in X, a_x \geq 0 \qandq \sum_{x \in X} a_x = 1 }.
}
One thus has for $a \in \La_X$, $\DiscMeasX{a}{X} \in \Mm_1^+(X)$. 

%%%
\paragraph{Discretized Wasserstein barycenter on $\Tt$. }

We first define approximate 1-D Wasserstein barycenters with an Eulerian discretization. The cumulative sum of $a \in \La_{\Tt}$ is 
\eq{
	\foralls t \in \Tt, \quad
	I( a )_t = \sum_{t' \leq t} a_{t'}.
}
The cumulative distribution is defined by approximating with sums and interpolation the formula~\eqref{eq-cumulative-defn}, for $\mu = \DiscMeasX{a}{\Tt} \in \Mm_1^+(\RR)$ 
\eq{
	\foralls t \in \RR, \quad \bar C_{ \mu }(t) = \text{Interp}( I(a) )( t ).
}	
Here, $\text{Interp} : \RR^{\Tt} \rightarrow \Cont{\RR}$ is an interpolation operator, that we take in the following to be piecewise linear.
We then define the approximate barycenter on $\Tt$ of measures $( \mu_i = \DiscMeasX{a_i}{\Tt} )_{i \in I} \in \Mm_1^+(\RR)^I$ denoted
\eq{
	\Bary{\Tt}( \mu_i,\la_i )_{i \in I} = \DiscMeasX{a^\star}{\Tt}
}
by applying formula~\eqref{eq-bary-1d-formula-deriv} on the grid $\Tt$, i.e.
\eq{
	\qwhereq
	\foralls t \in \Tt, \quad
	a^\star_t =  
	\frac{\d}{\d x}\pa{ \sum_{i \in I} \la_i \bar C_{\mu_i}^+(x) }^+( t ).
}
In practice, this formula is computed accurately by computing the inverse cumulative function on a uniform grid of $[0,1]$ of the same granularity as the spatial discretization, and computing the derivative with finite differences on this grid. 

%%%
\paragraph{Discretized Wasserstein barycenter on $\tilde\Gg$. }

One computes Eulerian barycenters on $\Om^d$ by computing 1-D barycenters of the marginals restricted to the grid $\tilde\Gg$. Indeed, we have for $\be \in \La_{\tGg}$, denoting $\nu = \DiscMeasX{\be}{\tGg}$, the disintegration formula on the grid
\eq{
	\foralls \th \in \Th, \:
	\nu^\th =  \DiscMeasX{\be_{\cdot,\th}}{\Tt}
	\qwhereq
	\be_{\cdot,\th} = ( \be_{(t,\th)} )_{t \in \Tt} \in \RR^{\Tt}.
}
The approximate barycenter on $\tGg$ of measures $( \nu_i = \DiscMeasX{\be_i}{\tGg} )_{i \in I} \in \bar\Mm_1^+(\Om^d)^I$ is thus
\eq{
	\Bary{\tGg}( \nu_i,\la_i )_{i \in I} = \DiscMeasX{\be^\star}{\tGg} = \nu^\star
}
\eq{
	\qwhereq
	\foralls \th \in \Th, \quad
	 (\nu^\star)^\th = 
	 \Bary{\Tt}( \nu_i^\th,\la_i )_{i \in I}.
}

%%%
\paragraph{Discrete Radon transform}

In the following, we investigate the use of the Fast Slant Stack Radon transform~\cite{Averbuch-slantstack}. It has the property to faithfully approximate the geometry of the Radon transform, i.e., it exactly computes integrals over 1-D rays for band limited functions. Note that other discretizations could be used as well, see for instance~\cite{Brady-Radon}. In the case of a 2-D Fast Slant Stack transform, the sampling grid $\tGg$ is recto-polar (so that $\tilde\Gg$ is in fact not an exactly equi-spaced grid, but we ignore this technicality here) and $|\Tt| = n$, $|\Th|=4n$. This Fast Slant Stack implements both the computation of the Radon transform and its adjoint with fast algorithms. These algorithms assume that the data is sampled from a band limited function, faithfully integrated using Shannon interpolation. This can thus result in negative values in the Radon transform, and in turn necessitates a careful implementation of the barycenter computation. 

We thus assume that we have at our disposal a discrete Radon transform (in our case the Fast Slant Stack), which is a linear map $\tilde R : \RR^\Gg \mapsto \RR^{\tGg}$, and also have access to its adjoint $\tilde R^* : \RR^{\tGg} \mapsto \RR^{\Gg}$. The Moore-Penrose pseudo-inverse 
\eq{
	{\tilde R}^+(\be) = ({\tilde R}^* {\tilde R})^{-1}{\tilde R}^*(\be) = 
	\uargmin{\al} \norm{{\tilde R} \al - \be} 
} 
is usually computed by a conjugate gradient descent. As reported in~\cite{Averbuch-slantstack}, it is possible to introduce a simple pre-con\-di\-tion\-ner for the Fast Slant Stack inversion that accelerates convergence of the conjugate descent, and is a major computational advantage for this approach.

This discrete Radon transform allows one to approximate the Radon transform of measures defined in~\eqref{eq-radon-measure} as
\eq{
	\foralls \al \in \RR^{\Gg}, \quad
	R( \DiscMeasX{ \al }{\Gg} ) \approx \DiscMeasX{ \tilde R(\al) }{\tGg}.
}
We leave for future work the theoretical analysis of this approximation when $\DiscMeasX{ \al }{\Gg} \rightarrow \mu$ and $(N,P)$ increases toward $+\infty$. 

% Although we do not give a more precise statement about this approximation, it should be understood typically as a weak-convergence of measures (or equivalently Wasserstein-distance convergence) of $\DiscMeasX{ \tilde R(\al) }{\tGg}$ toward $R(\mu)$ when $\DiscMeasX{ \al }{\Gg} \rightarrow \mu$ and $(N,P)$ increases toward $+\infty$. 

%%%
\paragraph{Approximated Radon Barycenters}

Making use of these discrete constructions (barycenters on $\tGg$ and Radon transform on $\Gg$), we are now ready to define the approximate Eulerian barycenter of measures supported on $\Gg$. 
We are thus given as input Eulerian discretized densities
\eq{
	\foralls i \in I, \quad 
	\mu_i = \DiscMeasX{\al_i}{\Gg}
	\qwhereq
	\al_i \in \RR^{\Gg}. 
}
The algorithm then computes the discretized Radon transform 
\eq{
	\foralls i \in I, \quad 
	\be_i = \tilde R( \al_i ) \in \RR^{\tGg}. 
}
For any $\la \in \La_I$, our Eulerian discretized Radon barycenter
\eq{
	\Bary{\Gg}^R( \mu_i, \la_i )_{i \in I} = \DiscMeasX{\al^\star}{\Gg}
	\qwhereq
	\choice{
		\al^\star = \tilde R^+ \be^\star, \\
		\DiscMeasX{\be^\star}{\tGg} = \Bary{\tGg}( \DiscMeasX{\be_i}{\tGg}, \la_i )_{i \in I}.
	} 
}
This barycenter is hence intended to approximate an element of $\Bary{\RR^d}^R(\mu_i)_{i \in I}$, with the constraint of being supported on $\Gg$. 



