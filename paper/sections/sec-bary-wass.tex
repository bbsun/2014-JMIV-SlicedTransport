% !TEX encoding = ISO-8859-16
%%%%%%%%%%%%%%%%%%%%%%%%%%%%%%%%%%%%%%%%%%%%%%%%%%%%%%%%%%%%%%
%%%%%%%%%%%%%%%%%%%%%%%%%%%%%%%%%%%%%%%%%%%%%%%%%%%%%%%%%%%%%%%%
\section{Wasserstein Distance}
\label{sec-bary-wass}


%%
\subsection{Optimal Transport}

For $(\mu_1,\mu_2) \in \Mm_1^+(\RR^d)^2$, we define the $L^2$-Wasserstein distance $\Wass{\RR^d}(\mu_1,\mu_2)^2$ to be equal to
\eql{\label{eq-dfn-wass-dist}
	\inf \enscond{ 
		\int_{\RR^d \times \RR^d} \norm{x_1-x_2}^2 \d\ga(x_1,x_2)
	}{
		{ \ga \in C(\mu_1,\mu_2) }
	}
}
where 
\eq{
	C(\mu_1,\mu_2) = \enscond{ \ga \in \Mm_1^+(\RR^d \times \RR^d) }{ \Pi_i \sharp \ga = \mu_i, \;  i=1,2 }
}
where $\Pi_1(x_1,x_2) = x_1$ and $\Pi_2(x_1,x_2)=x_2$. We refer to~\cite{Villani03} for more details regarding optimal transport and properties of the Wasserstein distance.

%%
\subsection{Wasserstein Barycenter on $\RR^d$}


Following~\cite{Carlier_wasserstein_barycenter}, we define the Wasserstein barycenter as a natural extension of the variational formula for barycenters in $\RR^d$.

\begin{defn}[Wasserstein barycenter]\label{defn-wass-baryc} Given $\la \in \La_I$ and $(\mu_i)_{i \in I} \in \Mm_1^+(\RR^d)^I$, we define
\eql{\label{eq-wass-bary}
	\Bary{\RR^d}^W(\mu_i,\la_i)_{i \in I} = \uargmin{\mu \in \Mm_1^+(\RR^d)} \sum_{i \in I} \la_i \Wass{\RR^d}( \mu_i,\mu )^2.
}
\end{defn}

Note that the variational problem is convex but it does not necessarily have a unique solution so that in general $\Bary{\RR^d}^W(\mu_i,\la_i)_{i \in I}$ is a (convex) set of measures. The solution can be shown to be unique (so that $\Bary{\RR^d}^W(\mu_i,\la_i)_{i \in I}$ is a singleton) if at least one of the $\mu_i$ does not give mass to so called ``small sets'' (sets of Hausdorff dimension strictly smaller than $d$), see~\cite{Carlier_wasserstein_barycenter}.  A typical example of non-uniqueness can be shown on two input measures, for which a barycenter can be computed from any coupling measure $\ga$ solving~\eqref{eq-dfn-wass-dist}, see~\cite{Carlier_wasserstein_barycenter}, Section~4. If the two input measures are finite sums of Dirac's masses, then~\eqref{eq-dfn-wass-dist} is a finite dimensional linear program, which in general can fail to have a unique solution. The (convex) set of solution to this linear program thus defines a set of barycenters. 




\if 0

For the sake of completeness, we recall the following Theorem of~\cite{Carlier_wasserstein_barycenter}, which shows that this barycenter can be computed as the projection in $\RR^d$ of a measure on $(\RR^d)^I$ solving a linear program. 


\begin{thm}[\cite{Carlier_wasserstein_barycenter}]\label{thm-multimarginal-wass}
	The solutions $\Bary{\RR^d}^W(\mu_i,\la_i)_{i \in I}$ of~\eqref{eq-wass-bary} can be written as $\be_\la \sharp \psi^\star$ 
	where $\psi^\star \in \Mm( (\RR^d)^I )$ solves
	\eql{\label{eq-linprog-bary-wass}
		\umin{ \psi \in \Ga(\mu_i)_{i \in I} }
			\int_{ (\RR^d)^I } K_\la(s) \d \psi( x ) , 
	}
	where the cost function is 
	\eq{
		\foralls x = (x_i)_{i \in I} \in \RR^I, \quad
			K_\la(x) = \sum_{i \in I} \la_i \abs{ x_i - \be_\la(x) }^2
	}
	and where the marginal constraint is defined as
	\eq{
		\Ga(\mu_i)_{i \in I} = \enscond{ \psi \in \Mm( (\RR^d)^I )
		}{
			\foralls i \in I,
			\quad 
			\Pi_i \sharp \psi = \mu_i
		}
	}
	\eq{
		\qwhereq \foralls i \in I, \; \foralls x \in (\RR^d)^I, 
		\quad 
		\Pi_i(x) = x_i \in \RR^d. 
	}
\end{thm}

\fi

It is proved in~\cite{Carlier_wasserstein_barycenter} that this barycenter can be computed as the projection in $\RR^d$ of a measure on $(\RR^d)^I$ solving a linear program. This theorem shows that, in the particular case where the input measures are discrete probability measures (i.e. sums of weighted Diracs) then the barycenter measures solving~\eqref{eq-wass-bary} are discrete probability measures, which can be computed by solving a finite dimensional linear program. Note that since in this case all the input measures do give mass to small sets, then the barycenter can be non-unique for some degenerate configurations of input Diracs. Also note that solving such a high dimensional linear program is intractable for imaging applications. This is one of the main motivations to introduce alternative definitions of barycenters of measures. 

The following proposition states some invariance properties of the Wasserstein barycenter with respect to translation, scaling, rotation and symmetry.

\begin{prop}\label{prop-invariance-W}
	We consider $\la \in \La_I$, $(\mu_i)_{i \in I} \in \Mm_1^+(\RR^d)^I$. 
	For all $(s,u) \in \RR^{+,*} \times \RR^d$, 
	\begin{align} \label{eq-prop-inv-1}
		\Bary{\RR^d}^W(\phi_{s,u} \sharp \mu_i,\la_i)_{i \in I} & = 
				\phi_{s,u} \sharp \Bary{\RR^d}^W(\mu_i,\la_i)_{i \in I},
	\end{align}
	and for all $\Phi \in \Oo(\RR^d)$, 
	\begin{align} \label{eq-prop-inv-rot}
		\Bary{\RR^d}^W( \Phi \sharp \mu_i,\la_i)_{i \in I} & = 
				\Phi \sharp \Bary{\RR^d}^W(\mu_i,\la_i)_{i \in I}.
	\end{align}	
	In particular, one has
	\begin{align} \label{eq-prop-inv-rad}
		&\foralls i \in I, \mu_i \in \Rad(\RR^d) \\
			& \Rightarrow \Bary{\RR^d}^W(\mu_i,\la_i)_{i \in I} \subset \Rad(\RR^d),
	\end{align}
	and also
	\begin{align} \label{eq-prop-inv-cent}
		&\foralls i \in I, \mu_i \in \Cent(\RR^d) \\
			& \Rightarrow \Bary{\RR^d}^W(\mu_i,\la_i)_{i \in I} \subset \Cent(\RR^d).
	\end{align}	
\end{prop}

The proof of this proposition, as well as all the other proofs of this section, can be found in Appendix~\ref{sec-appendix-wass}.
The following proposition shows that the Wasserstein barycenter of translated and scaled copies of a given measure is also a translated and scaled copy.

\begin{prop}\label{prop-invariance-W-bis}
	We consider $\la \in \La_I$, $\mu  \in \Mm_1^+(\RR^d)$. 	
	For all $(s_i,u_i)_{i \in I} \in (\RR^{+,*} \times \RR^d)^I$, 
	\begin{align} \label{eq-prop-inv-2}			
		\phi_{s^\star,u^\star} \sharp \mu \in \Bary{\RR^d}^W(\phi_{s_i,u_i} \sharp \mu,\la_i)_{i \in I}, 
		 \qwhereq
	\end{align}
	\eql{\label{prop-invariance-W-bis-formula}
		s^\star = \pa{\sum_{i \in I} \la_i s_i^{-1}}^{-1}
		\qandq
		u^\star = \frac{ \sum_{i \in I} \la_i s_i^{-1} u_i }{ \sum_{i \in I} \la_i s_i^{-1} }.
	}
\end{prop}


%%
\subsection{Wasserstein Barycenter on $\RR$}

The following result shows that it is possible to compute a Wasserstein barycenter measure solving~\eqref{eq-wass-bary} in the 1-D case, with a close form expression. Note that if all the input measures contain Dirac atoms, the barycenter is not necessarily unique. 


\begin{prop}\label{prop-bary-1d-pushfwd}
	Let $\mu \in \Mm_1^+(\RR)$ be absolutely continuous with respect to the Lebesgue measure (i.e., such that $\mu$ has a density), and $(\mu_i)_{i \in I} \in \Mm_1^+(\RR)^I$. Denoting $T_i$ such that $T_i \sharp \mu = \mu_i$ the optimal transport between $\mu$ and $\mu_i$ (which is unique), then 
	\eql{\label{eq-bary-1d-pushfwd}
		\mu^\star = \pa{ \sum_{i \in I}  \la_i T_i } \sharp \mu
	}
	is a barycenter measure solving~\eqref{eq-wass-bary}, i.e. $\mu^\star \in \Bary{\RR}^W(\mu_i,\la_i)_{i \in I}$.
\end{prop}


For $\mu \in \Mm(\RR)$, we write the cumulative function as 
\eql{\label{eq-cumulative-defn}
	\foralls t \in \RR, \quad
	C_\mu(t) = \mu(]-\infty,t]).
}
As for any non-decreasing function $f : \RR \rightarrow \RR$, one can define its pseudo inverse 
\eql{\label{eq-cumulative-pseudoinv-defn}
	\foralls t \in \RR, \quad
	f^+(t) = \inf
	\enscond{s \in \RR}{ f(s) \geq t }.
}
The following corollary shows that 1-D barycenters can be computed almost in closed form using inverse cumulative functions.


\begin{cor}\label{prop-bary-1d}
Given $(\mu_i)_{i \in I} \in \Mm_1^+(\RR)^I$, and $\la \in \La_I$. Then 
\eql{\label{eq-bary-1d-formula-deriv}
	\mu^\star = \frac{\d}{\d t}\pa{ \sum_{i \in I} \la_i C_{\mu_i}^+(t) }^+,
} 
where the derivative should be interpreted in the sense of distribution, satisfies 
$\mu^\star \in \Bary{\RR^d}^W(\mu_i,\la_i)_{i \in I}$, i.e., is a barycenter measure.
In particular, it satisfies
\eq{
	C_{\mu^\star}^+ = \sum_{i \in I} \la_i C_{\mu_i}^+.
}
\end{cor}



%%
\subsection{Wasserstein Barycenter on $\Om^d$}

We extend 1-D Wasserstein barycenters to barycenters of measures on $\Om^d$ by essentially computing the barycenter along the $t$ variable only. For this to be feasible, we restrict our attention to measures in $\bar\Mm_1^+(\Om^d)$ having normalized conditional densities along the $t$ variable. 


\begin{defn}[Wasserstein Barycenter on $\Om^d$]
Given $(\nu_i)_{i \in I} \in  \bar\Mm_1^+(\Om^d)^I$ and $\la \in \La_I$, we define the barycenters as
\eq{
	\nu \in \Bary{\Om^d}^W(\nu_i,\la_i)_{i \in I} \in \bar\Mm_1^+(\Om^d)
} 
by, for almost all  $\th \in \Sph$, 
\eq{
	\nu^{\th} \in 
	\Bary{\RR}^W(\nu_i^\th,\la_i)_{i \in I}.
}
\end{defn}

Considering the following extension of the Wasserstein distance to $\bar\Mm_1^+(\Om^d)$ by integrating 1-D Wasserstein distances, $\foralls (\nu_1,\nu_2) \in \bar\Mm_1^+(\Om^d)^2$, 
\eq{
	\Wass{\Om^d}(\nu_1,\nu_2)^2 
	= \int_{\Sph} \Wass{\RR}( \nu_1^\th, \nu_2^\th )^2 \d \th,
}	
we have the following characterization of the Wasserstein barycenter on $\Om^d$.

\begin{prop}\label{prop-bary-omegad-variational}
One has
\eql{\label{eq-bary-omegad-variational}
	\Bary{\Om^d}^W(\nu_i,\la_i)_{i \in I} =
	\uargmin{\nu \in \bar\Mm_1^+(\Om^d)} 
	\sum_{i \in I} \la_i \Wass{\Om^d}(\nu_i,\nu)^2.
}
\end{prop}

The following proposition exposes some useful properties of barycenters in $\Om^d$.

\begin{prop}\label{prop-bary-omd}
	If $\nu \in \bar\Mm_{1}^+(\Om^d)$, then $\psi_{s,u} \sharp \nu \in \bar\Mm_{1}^+(\Om^d)$, and
	\begin{align}
		\label{propBar1} 
		\Bary{\Om^d}^W(\psi_{s,u} \sharp \nu_i,\la_i)_{i \in I} &= \psi_{s,u} \sharp \Bary{\Om^d}^W(\nu_i, \la_i)_{i \in I} \\
		\label{propBar2} 
		\Bary{\Om^d}^W(\psi_{s_i,u_i} \sharp \nu,\la_i)_{i \in I} &= \psi_{s^\star,u^\star} \sharp \nu 
		% \Bary{\Om^d}^W(\nu, \la_i)_{i \in I}
	\end{align}
	where $s^\star$ and $u^\star$ are defined in~\eqref{prop-invariance-W-bis-formula}. 
\end{prop}


