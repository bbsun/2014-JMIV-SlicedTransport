\section{Proofs of Section~\ref{sec-bary-radon}}
\label{sec-appendix-radon}

\begin{proof}[Proof of Proposition~\ref{prop-radon-pushforward}]
For all $g \in \Cont{\Om^d}$, one has 
\begin{align*}
&\int_{\Sph}\int_{\RR} g(t, \th) \d(R(\mu)^\th)(t) \d\th = \int_{\Om^d}g(t, \th)\d(R(\mu))(t, \th) \\
	&\qquad = \int_{\RR^d}(R^*g)(x)\d\mu(x) \\
	&\qquad = \int_{\RR^d}\int_{\Sph} g(P_\th(x),\th) \d\th \d\mu(x)\\
	&\qquad = \int_{\Sph} \int_{\RR} g(y,\th) \d(P_\th \sharp \mu)(y) \d\th.
\end{align*}
\end{proof}


\begin{proof}[Proof of Lemma~\ref{lem-invariances}]	
		%%%%% 
		\if 0 %%% ALTERNATE %%%
		\noindent\textit{Proof of~\eqref{propR}:} 
		Using Proposition~\ref{prop-radon-pushforward}, one has, for all $\th \in \Sph$, 
		\eq{
			R( \phi_{s,u} \sharp \mu )^\th = P_\th \sharp (\phi_{s,u} \sharp \mu) 
			= (P_\th \circ \phi_{s,u}) \sharp \mu.
		}
		Notice that
		\eq{
			P_\th \circ \phi_{s,u}(x) = 
			s\dotp{x}{\th} + \dotp{u}{\th} = 
			\phi_{s,\dotp{u}{\th}} \circ P_\th.
		}
		where here we used $\phi_{s,\dotp{u}{\th}} : \RR \rightarrow \RR$.
		Hence
		\eq{
			R( \phi_{s,u} \sharp \mu )^\th = 
			\phi_{s,\dotp{u}{\th}} \sharp (P_\th \sharp \mu) = 
			\phi_{s,\dotp{u}{\th}} \sharp R(\mu)^\th
		}
		which is equivalent to~\eqref{propR}.		\\
		\fi
		%%%%%
		\noindent\textit{Proof of~\eqref{propR}:} 
		For all $g \in \Cont{\Om^d}$, one has 
		\begin{align*}
			\int_{\RR^d} g \d[ R (\phi_{s,u} \sharp \mu)] 
				&= \int_{\RR^d} R^*(g) \d[ \phi_{s,u} \sharp \mu] \\
%				& = \int_{\RR^d} \int_{\Sph} g(\dotp{x}{\th},\th) \d\th \d[ \phi_{s,u} \sharp \mu](x) \\
				&= \int_{\RR^d} \int_{\Sph} g(\dotp{sx+u}{\th},\th) \d\th\d\mu(x) \\
%				&= \int_{\RR^d} \int_{\Sph} g(s\dotp{x}{\th}+\dotp{u}{\th},\th) \d\th\d\mu(x) \\
				&= \int_{\RR^d} \int_{\Sph} (g \circ \psi_{s,u})(\dotp{x}{\th},\th) \d\th\d\mu(x) \\
%				&= \int_{\RR^d} R^* (g \circ \psi_{s,u}) \d\mu(x) \\
				&= \int_{\RR^d} (g \circ \psi_{s,u}) \d[R(\mu)] \\
				& = \int_{\RR^d} g \d[ \psi_{s,u} \sharp R(\mu)]
		\end{align*}		
		%%%%%%
		\if 0 %%% WRONG %%%
		for all $g \in \Cont{\Om^d}$, one has 
		\begin{align*}
		& \int_{\Om^d} g(t,\th) \d(R(\phi_{s,u}\sharp \mu))  = \int_{\RR^d}(R^* g)(s x + u) \d\mu(x) \\
			& \quad = \int_{\RR^d} \int_{\dotp{s x + u}{\th}=t} g(t, \th) \d t \d\th \d\mu(x) \\
			& \quad = \int_{\RR^d}\int_{\dotp{x}{\th}=\frac{t-\dotp{u}{\th}}{s}} g(t, \th) \d t \d\th \d\mu(x)\\
			& \quad = \int_{\RR^d}\int_{\dotp{x}{\th}=t'} g(s t'+\dotp{u}{\th}, \th) s \d t' \d\th \d\mu(x)
		\end{align*}
		\fi
		%%%%%
		\if 0
		\noindent\textit{Proof of~\eqref{propRm}:} 			
		Using~\eqref{eq-pushfwd-density}, and applying~\eqref{propR} to a measure $\d\mu(x) = f(x) \d x$ with density $f$, one has the following invariance of the Radon transform of function 
		\eq{
		\begin{align*}
			R( f \circ \phi_{s,u} ) &= s^{1-d} R(f) \circ \psi_{s,u}\\
			& = 
		\end{align*}		
		}
		Thus for all $f \in \Cont{\RR^d}$, one has
		\begin{align*}
		\int_{\RR^d} f \d[R^*(\psi_{s,u}\sharp \nu)]
			& = \int_{\Om^d} (R f) \circ \psi_{s,u} \d\nu \\
			& = s^{d-1} \int_{\Om^d} R (f \circ \phi_{s,u} ) \d\nu \\
			& = s^{d-1} \int_{\Om^d} f \circ \phi_{s,u}  \d (R^*\nu) \\
			& = \int_{\Om^d} f   \d[s^{d-1} \cdot \phi_{s,u}\sharp  R^*(\nu)]
		\end{align*}
		\fi 		
		%%%%%%%
		\noindent\textit{Proof of \eqref{propRp}}: 		
		First we notice, using \eqref{eq-radon}, that 
			\begin{align*} 
				& R( f \circ \phi_{s,u} )(t,\th) 
				= \int_{\R^{d-1} } f\pa{ s(t\theta + U_\th \ga) + u } \d\ga \\
				& \qquad = \int_{\R^{d-1} } f\pa{ st\th + U_\th s\ga + \dotp{u}{\th}\th + U_\th(U_\th)^{T}u } \d\ga \\
				& \qquad = \int_{\R^{d-1} } f\pa{ (st+\dotp{u}{\th})\th + U_\th (s\ga + (U_\th)^{T}u)} \d\ga \\
				& \qquad = s^{1-d}  \int_{\R^{d-1} } f\pa{ \psi_{s,u} (t,\th) \th + U_\th \ga' } \d \ga'
			%		= s^{1-d} R f (t',\th)\\
			%	& \qquad = s^{1-d} R(f) \circ \psi_{s,u} (t,\th).
			\end{align*}
			which proves
			\eql{\label{eq-invariance-R}
				R( f \circ \phi_{s,u} ) = s^{1-d} R(f) \circ \psi_{s,u}
			}
			%%% 
		\if  0 %% OLD
			Note that using~\eqref{eq-pushfwd-density}, and applying~\eqref{propR} to a measure $\d\mu(x) = f(x) \d x$ with density $f$, one has the following invariance of the Radon transform of function 
			\eql{\label{eq-invariance-R}
				R( f \circ \phi_{s,u} ) = s^{1-d} R(f) \circ \psi_{s,u}.
			}		
		\fi %%%
			We write $H = (R^*R)^{-1}$ the filtering operator with kernel $h^+$. One has, for smooth functions $f \in \Ss(\RR^d)$, denoting $\Ff(f)=\hat f$, 
			\begin{align*}
				\Ff( H(f \circ \phi_{s,u}) ) &= c^{-1}\norm{\om}^{1-d} \hat f(s\om) e^{-\imath \dotp{\om}{u}}, \\
				\Ff( H(f) \circ \phi_{s,u} ) &= c^{-1}\norm{s\om}^{1-d} \hat f(s\om) e^{-\imath \dotp{\om}{u}}, 
			\end{align*}
			and hence
			\eql{\label{eq-H-invariance}
				H(f) \circ \phi_{s,u} = 
				s^{1-d} H(f \circ \phi_{s,u}).
			}
			This shows, using~\eqref{eq-invariance-R} and~\eqref{eq-H-invariance} that for all $f \in \Dd(\RR^d)$, 
			\begin{align*}
				\int_{\RR^d} f \d[ R^+( \psi_{s,u} \sharp \nu ) ] 
				&= \int_{\RR^d} (RHf) \circ \psi_{s,u}  \d \nu \\ 
				&= s^{d-1} \int_{\RR^d} R(H(f)  \circ \phi_{s,u})  \d \nu \\
				&= \int_{\RR^d} RH(f  \circ \phi_{s,u})  \d \nu \\
				&= \int_{\RR^d} f  \d[ \phi_{s,u} \sharp  R^+(\nu) ] \\
			\end{align*}
		%%%%%%%
		\noindent\textit{Proof of \eqref{PropRot}}: the proof is similar to the one of~\eqref{propR}.
\end{proof}

\begin{proof}[Proof of Proposition~\ref{prop:InvarianceHolds}]
		Using Lemma~\ref{lem-invariances}, one has
		\begin{align*}
		\Bary{\RR^d}^R(\phi_{s,u} \sharp \mu_i,\la_i)_{i \in I} 
			&= R^+ \Bary{\Om^d}^W(R(\phi_{s,u}\sharp \mu_i), \la_i)_{i \in I}\\
			&= R^+ \Bary{\Om^d}^W(\psi_{s,u}\sharp (R(\mu_i)), \la_i)_{i \in I}\\
			&= R^+ \psi_{s,u} \sharp \Bary{\Om^d}^W(R(\mu_i), \la_i)_{i \in I}\\
			&= \phi_{s,u} \sharp R^+ \Bary{\Om^d}^W(R(\mu_i), \la_i)_{i \in I}\\
			&= \phi_{s,u} \sharp \Bary{\RR^d}^R(\mu_i,\la_i)_{i \in I}.
		\end{align*}
		which proves~\eqref{eq-prop-inv-1} for $\Bary{\RR^d}^R$.
		Property~\eqref{eq-prop-inv-rot} for $\Bary{\RR^d}^R$ is proved similarly using~\eqref{PropRot}.
\end{proof}


		
\begin{proof}[Proof of Proposition~\ref{prop:InvarianceHoldsBis}]
		One has
		\begin{align*}
		\Bary{\RR^d}^R(\phi_{s_i,u_i} \sharp \mu,\la_i)_{i \in I} 
			&= R^+ \Bary{\Om^d}^W(R(\phi_{s_i,u_i}\sharp \mu), \la_i)_{i \in I}\\
			&= R^+ \Bary{\Om^d}^W( \psi_{s_i,u_i}\sharp R(\mu), \la_i)_{i \in I}\\
			&= R^+  \psi_{s^\star,u^\star} \sharp \Bary{\Om^d}^W(R(\mu), \la_i)_{i \in I}\\
			&= \phi_{s^\star,u^\star} \sharp R^+ \Bary{\Om^d}^W(R(\mu), \la_i)_{i \in I}\\
			&= \phi_{s^\star,u^\star} \sharp \Bary{\RR^d}^R(\mu,\la_i)_{i \in I}, % \\
		\end{align*}
		which proves \eqref{eq-prop-inv-2} for $\Bary{\RR^d}^R$.
\end{proof}

